\documentclass{article}
\usepackage[utf8]{inputenc}
\usepackage{amsmath,amsthm,amssymb}
\usepackage{amsfonts}
\usepackage{arydshln}
\usepackage{enumitem}
\usepackage{float}
\usepackage{graphicx}
\usepackage{hyperref}
\usepackage{listings}
\usepackage{makecell}
\usepackage[margin=0.5in]{geometry}
\usepackage{multicol}
\usepackage{subcaption}
\usepackage{wrapfig}
\allowdisplaybreaks
\newtheorem{theorem}{Theorem}
\newtheorem{lemma}{Lemma}

\title{{\large Math 465}\\ Homework 01}
\author{Bridgette Delight}
\date{\today}

\begin{document}

\maketitle

\section{}
By following the steps of the proof of Theorem 1.1 and the proof of The Contraction Mapping Theorem, prove the following facts:

\begin{enumerate}[label = (\alph*)]
    \item the function $f(x)= \frac{1}{\sqrt{2}+0.5x}-1.5x$ has exactly one root in the interval $\left[0, \frac{1}{2}  \right]$,
    \item the sequence of approximations $\left\{ s_k \right\}^{\infty}_{k=0}$ defined recursively by 
    \begin{equation}\label{cases}
    \begin{cases}
    s_0 \in \left[0, \frac{1}{2} \right] - \text{arbitrary}\\
    s_{k+1} = \frac{2}{3\sqrt{2}+1.5s_k}, &k=0,1,2,\dots
    \end{cases}
    \end{equation}converges to the root.
\end{enumerate}


\vspace{10mm}

\section{}
Take $s_0 = \frac{1}{\sqrt{6}}$ in (\ref{cases}) and write a code that you can use to compute the first 10000 elements of the sequence $\left\{ s_k \right\}^{\infty}_{k=0}$.
\vspace{10mm}


\section{}
Determine a formula for the exact value of the root $\xi \in \left[0,\frac{1}{2} \right]$ of the function $f$.
\vspace{10mm}


\section{}
Run your code and present a table that displays the iteration number $k \in {1,2,\dots,15}$ in the first column, the corresponding approximations $s_k$ in the second column, and the corresponding errors of the approximations in the third column.
\vspace{10mm}

\section{}
Present a figure showing the approximations $s_k$ (vertical axis) versus the iteration index $k$ (horizontal axis) for $k=1,2,\dots,15$.
\vspace{10mm}

\section{}
Present a figure showing the errors $|s_k - \xi|$ of the approximations (vertical axis) versus the iteration index $k$ (horizontal axis) for $k=1,2,\dots,15$.
\vspace{10mm}


\section{}
For any $s_0 \in \left[0, \frac{1}{2} \right]$, derive an upper bound on the number of iteration $k$ required to ensure that the $k^{th}$ iterate $s_k$ is correct to six decimal digits. Shows all steps of your derivation. Write down the value of the upper bound in case $s_0 = \frac{1}{\sqrt{6}}$.
\vspace{10mm}

\section{}
Run your code with $s_0 \in \left[0,\frac{1}{\sqrt{6}} \right]$ and find the index $i$ of the first iterate $s_i$ that is correct to six decimal digits. Compare this to your answer to question 7. Is this what you have expected? Why?
\vspace{10mm}





\end{document}