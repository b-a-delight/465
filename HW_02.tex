\documentclass{article}
\usepackage[utf8]{inputenc}
\usepackage{amsmath,amsthm,amssymb}
\usepackage{amsfonts}
\usepackage{arydshln}
\usepackage{datetime}
\usepackage{enumitem}
\usepackage{float}
\usepackage{graphicx}
\usepackage{hyperref}
\usepackage{listings}
\usepackage{makecell}
\usepackage[margin=0.75in]{geometry}
\usepackage{multicol}
\usepackage{subcaption}
\usepackage{wrapfig}
\allowdisplaybreaks
\newtheorem{theorem}{Theorem}
\newtheorem{lemma}{Lemma}

\mmddyydate


\usepackage{fancyhdr}
\pagestyle{fancy}
\fancyhf{}
\fancyhead[L]{Bridgette Delight}
\fancyhead[C]{\large \textbf{Math 465 - Homework 02 - \today}}
\fancyhead[R]{pg. \thepage}
\renewcommand{\headrulewidth}{2pt}

\title{{\large Math 465}\\ Homework 0X}
\author{Bridgette Delight}
\date{\today}

\begin{document}

%\maketitle

\section{}
Consider the functions
\begin{equation}\label{eq:1a}
    f(x) = 3\pi - x - 0.01 \sin(x), \qquad \text{for }x \in \mathbb{R}
\end{equation}

\begin{equation}\label{eq:1b}
    g_1(x) = 3\pi - 0.01 \sin(x), \qquad \text{for }x \in \mathbb{R}
\end{equation}

\begin{equation}\label{eq:1c}
    g_2(x) =  x - \frac{f(x)}{f'(x)}, \qquad \text{for }x \in \mathbb{R} \text{ such that }f'(x) \ne 0,
\end{equation}

and show that the following statements are equivalent
$$f(\xi) = 0 \Longleftrightarrow g_1(\xi)=\xi g_2(\xi) = \xi.$$

\vspace{10mm}

\begin{align*}
    \text{First prove that:}&\\
    f(\xi) =0 & \Longleftrightarrow g_1(\xi)=\xi\\
    f(\xi) =0 &\quad 0 = 3\pi - \xi - 0.01\sin(\xi)\\
    g_1(\xi)=\xi & \quad \xi = 3\pi - 0.01\sin(\xi)\\
    0 &= 3\pi - \xi - 0.01\sin(\xi)\\
    \text{Therefore,}&\\
    f(\xi) =0 & \Longleftrightarrow g_1(\xi)=\xi\\
    \text{Now prove that,}&\\
     g_1(\xi)&= g_2(\xi) = \xi\\
     g_1(\xi)=\xi & \quad \xi = 3\pi - 0.01\sin(\xi)\\
     0 &= 3\pi - \xi - 0.01\sin(\xi)\\
     g_2(\xi)=\xi & \quad \xi= \xi - \frac{f(\xi)}{f'(\xi)}\\
     \xi &= \xi - \frac{3\pi - \xi - 0.01\sin(\xi)}{-1-0.01\cos(\xi)}\\
     0 &= - \frac{3\pi - \xi - 0.01\sin(\xi)}{-1-0.01\cos(\xi)}\\
     \text{Since you cannot divide by zero, that means the}&\\
     \text{numerator is equal to zero, and since the }&\\
     \text{numerator is equal to $g_1(\xi)=\xi$, then,}\\
     &g_1(\xi) = g_2(\xi) = \xi \\
    \text{Therefore,}&\\
    &f(\xi) = 0 \Longleftrightarrow g_1(\xi)=\xi g_2(\xi) = \xi
\end{align*}



\section{}
Show that $\xi = 3\pi$ is a root of $f$ and determine $g_2'(\xi)$.
\vspace{10mm}

\begin{align*}
    f(x) &= 3\pi - x - 0.01 \sin(x)\\
    f(3\pi) &= 3\pi - 3\pi - 0.01 \sin(3\pi)\\
    &= 0 - 0.01 \sin(3\pi)\\
    &=- 0-0 = 0\\
    \text{Therefore, $3\pi$ is a root of $f$.}&\\
    g_2(\xi) &= \xi - \frac{3\pi - \xi - 0.01\sin(\xi)}{-1-0.01\cos(\xi)}\\
    g_2'(\xi)&= 1 - \frac{(-1-0.01\cos(\xi)(-1-0.01\cos(\xi))-(3\pi - \xi - 0.01\sin(\xi))(0.01\sin(\xi))} {(-1-0.01\cos(\xi))^2}\\
    &= 1-\frac{(-1-0.01\cos(\xi))^2- (0.03\pi\sin(\xi)-0.01\xi\sin(\xi)- (0.01\sin(\xi))^2 )}{(-1-0.01\cos(\xi))^2}\\
    &=1- \frac{(-1-0.01\cos(\xi))^2- \sin(\xi)(0.03\pi-0.01\xi - (0.01\sin(\xi)) )}{(-1-0.01\cos(\xi))^2}\\
    g_2'(3\pi)&= 1- \frac{(-1-0.01\cos(3\pi))^2- \sin(3\pi)(0.03\pi-0.01\pi - (0.01\sin(\pi)) )}{(-1-0.01\cos(\pi))^2}\\
    &= 1- \frac{(-1-0.01(1))^2- 0(0.03\pi-0.01\pi - (0.01\cdot 0) )}{(-1-0.01(1))^2}\\
    &= 1- \frac{(-1-0.01(1))^2}{(-1-0.01(1))^2}= 1 - 1\\
    &= 0\\
\end{align*}


\section{}
Determine $L_1 \in (0,1)$such that $\underset{x \in \mathbb{R}}{max}|g_1'(x)|\le L_1$
\vspace{10mm}

\begin{align*}
    g_1'(x) &= -0.01\cos(x)\\
    |g_1'(x)| &= 0.01\cos(x)\\
    \underset{x \in \mathbb{R}}{max}|g_1'(x)| &= 0.01 \cdot 1 = 0.01\\
    L_1 &\ge 0.01\\
    \text{So,}&\\
    L_1 &= 0.01 \in (0,1)\\
\end{align*}

\section{}
Show that $3\pi$ is a unique root of the function $f:\mathbb{R} \xrightarrow{} \mathbb{R}$ defined by equation \ref{eq:1a}.
\vspace{10mm}

Theorem 1.3, the Contraction Mapping Theorem,\\
Assume:
\begin{enumerate}
      \item $g: [a,b] \to [a,b]$ is continuous
    \item $g$ is a contraction \\
    Then $g$ has a unique fixed point $\xi \in [a,b]$.
\end{enumerate}

Since $f: \mathbb{R}\to \mathbb{R}$ and $3\pi-x-0.01 \sin(x)$ is continuous then Assumption 1. is met.\par
For Assumption 2. the contraction condition can be replaced by a stronger condition.

\begin{align*}
    g_1-\xi  &= |g(x)-g_1(y)| \le L|x-y|\\
    L &\ge \frac{|g(x)-g_1(y)|}{|x-y|}\\
    L &\ge \bigg |\frac{g(x)-g_1(y)}{x-y} \bigg| \\
    L &\ge |g_1'(x)|\\
    L &\ge max|g_1'(x)|\\
    \text{from Problem 3 we know that,}&\\
    L &\ge 0.01\\
\end{align*}
So, Assumption 2. has been met. As all assumptions have been met then $g$ has a unique fixed point. Since $g_1(\xi)=\xi$ is unique and $g(\xi)t > f(\xi)=0$ then, $f(\xi)=0$ is unique. So, $f$ has a unique root $3\pi$.

\section{}
Write a code that you can use to compute the first $k_{max}+1$ approximations $x_k$ and $y_k$ to $3\pi$ from the following two algorithms

\begin{equation}\label{eq:4a}
    \begin{cases}
    x_0 = 100,\\
    x_{k+1}= g_1(x_k), & k=0,1,\dots , k_{max}-1\\
    \end{cases}
\end{equation}

and 

\begin{equation}\label{eq:4b}
    \begin{cases}
    y_0 = 100,\\
    y_{k+1}= g_2(y_k), & k=0,1,\dots , k_{max}-1\\
    \end{cases}
\end{equation}

where $k_{max}$ is an arbitrary positive integer and $g_1$ and $g_2$ are defined by equations \ref{eq:1b} and \ref{eq:1c}, respectively. 

\vspace{10mm}

\begin{verbatim}
import numpy as np
import pandas as pd
import math

def g1(n):
    sequence = [0, 100]
    for index in range(n):
        sequence.append(3*math.pi-0.01*math.sin(sequence[-1]))
    return sequence
    
def g2(n):
    sequence = [0, 100]
    for index in range(n):
        sequence.append(sequence[-1]-(3*math.pi-sequence[-1]-.01*math.sin(sequence[-1]))/(-1-.01*math.cos(sequence[-1])))
    return sequence
    
df = pd.DataFrame(g1(10))

df = df.rename(columns = {0: 'x'})
df['y'] = g2(10)

df = df.iloc[1:]
df = df.reset_index(drop=True)

df['x-Error'] = df.apply(lambda row : abs(row['x']-(3*math.pi)), axis = 1)
df['y-Error'] = df.apply(lambda row : abs(row['y']-(3*math.pi)), axis = 1)

df   

def z(n,df):
    sequence = [0, df.loc[2,'x']]
    for index in range(n):
       sequence.append(sequence[-1]-(3*math.pi-sequence[-1]-.01*math.sin(sequence[-1]))/(-1-.01*math.cos(sequence[-1])))
    return sequence

dfz = pd.DataFrame(z(2,df))

dfz = dfz.rename(columns = {0: 'z'})


dfz = dfz.iloc[1:]
dfz = dfz.reset_index(drop=True)

dfz['z-Error'] = dfz.apply(lambda row : abs(row['z']-(3*math.pi)), axis = 1)

dfz
\end{verbatim}


\section{}
Run your code for $k_{max}=10$ and present a table (Table 1) displaying the iteration number $k \in \{1,2,\dots,10\}$ in the first column, the corresponding approximations $x_k$ defined by equation \ref{eq:4a} in the second column, $y_k$ defined by \ref{eq:4b} in the third column, the corresponding errors of $x_k$ in the fourth column, and the errors of $y_k$ in the fifth column. Use {\fontfamily{qcr}\selectfont \textbf{format long e}}.
\vspace{10mm}

\begin{table}[H]
    \centering
    \begin{tabular}{|c|c|c|c|c|}
    \Xhline{1 pt}
    \textbf{k} & $x_k$& $y_k$& \makecell{errors\\$x_k $ }  & \makecell{errors \\$y_k$}  \\
    \Xhline{2 pt}
    \textbf{1} &9.429841617180477&10.20416802284721 &5.063656411097739e-03& 7.793900620778338e-01 \\
    \Xhline{1 pt}
    \textbf{2} &9.424828597117099&	9.426272920120425& 5.063634771929060e-05 & 1.494959351045821e-03\\
    \Xhline{1 pt}
    \textbf{3} &9.424778467132857&9.424777960780629&5.063634773705417e-07& 1.124966786392179e-11  \\
    \Xhline{1 pt}
    \textbf{4} &9.424777965833014&	9.424777960769379& 5.063634844759690e-09 & 0.000000000000000e+00 \\
    \Xhline{1 pt}
    \textbf{5} &9.424777960820016&	9.424777960769379 &5.063682806394354e-11 & 0.000000000000000e+00 \\
    \Xhline{1 pt}
    \textbf{6} &9.424777960769886&	9.424777960769379& 5.062616992290714e-13  & 0.000000000000000e+00 \\
    \Xhline{1 pt}
    \textbf{7} &9.424777960769385&9.424777960769379&5.329070518200751e-15& 0.000000000000000e+00 \\
    \Xhline{1 pt}
    \textbf{8} &9.424777960769379&	9.424777960769379& 0.000000000000000e+00 & 0.000000000000000e+00 \\
    \Xhline{1 pt}
    \textbf{9} &9.424777960769379&	9.424777960769379& 0.000000000000000e+00&  0.000000000000000e+00\\
     \Xhline{1 pt}
    \textbf{10} &9.424777960769379&	9.424777960769379& 0.000000000000000e+00&  0.000000000000000e+00\\
    \Xhline{1 pt}
    \end{tabular}
    \caption{Question 6}
    \label{tab:my_label}
\end{table}
 	

\section{}
Run your code for $k_{max}=2$ and compute approximations $z_k$ defined by

\begin{equation}\label{eq:7}
    \begin{cases}
    z_0 = x_2,\\
    z_{k+1}=g_2(z_k), & k=0,1,\dots , k_{max}-1,
    \end{cases}
\end{equation}

to present a table (Table 2) displaying the iteration number $k \ in \{0,1,2\}$ in the first column,the corresponding approximations $z_k$ defined by \ref{eq:7} in the second  column, and the corresponding errors of $z_k$ in the third column. Use {\fontfamily{qcr}\selectfont \textbf{format long e}}.
\vspace{10mm}

\begin{table}[H]
    \centering
    \begin{tabular}{|c|c|c|}
    \Xhline{1pt}
    \textbf{k} &$z_k$    & \makecell{errors \\ $z_k$}  \\
    \Xhline{2 pt}
    \textbf{0}   &9.424828597117099 & 5.063634771929060e-05\\
    \Xhline{1 pt}
    \textbf{1}   & 9.424777960769379& 0.000000000000000e+00\\
    \Xhline{1 pt}
    \textbf{2}   & 9.424777960769379 & 0.000000000000000e+00\\
    \Xhline{1 pt}
    \end{tabular}
    \caption{Question 7}
    \label{tab:q7}
\end{table}



\end{document}