\documentclass{article}
\usepackage[utf8]{inputenc}
\usepackage{amsmath,amsthm,amssymb}
\usepackage{amsfonts}
\usepackage{arydshln}
\usepackage{enumitem}
\usepackage{float}
\usepackage{graphicx}
\usepackage{hyperref}
\usepackage{listings}
\usepackage{makecell}
\usepackage[margin=0.75in]{geometry}
\usepackage{multicol}
\usepackage{subcaption}
\usepackage{wrapfig}
\allowdisplaybreaks
\newtheorem{theorem}{Theorem}
\newtheorem{lemma}{Lemma}

\usepackage{fancyhdr}
\pagestyle{fancy}
\fancyhf{}
\fancyhead[L]{Bridgette Delight}
\fancyhead[C]{\large \textbf{Math 465 - Homework 02}}
\fancyhead[R]{\thepage}
\renewcommand{\headrulewidth}{2pt}

\title{{\large Math 465}\\ Homework 0X}
\author{Bridgette Delight}
\date{\today}

\begin{document}

%\maketitle

\section{}
Consider the functions
\begin{equation}\label{eq:1a}
    f(x) = 3\pi - x - 0.01 \sin(x), \qquad \text{for }x \in \mathbb{R}
\end{equation}

\begin{equation}\label{eq:1b}
    g_1(x) = 3\pi - 0.01 \sin(x), \qquad \text{for }x \in \mathbb{R}
\end{equation}

\begin{equation}\label{eq:1c}
    g_2(x) =  x - \frac{f(x)}{f'(x)}, \qquad \text{for }x \in \mathbb{R} \text{ such that }f'(x) \ne 0,
\end{equation}

and show that the following statements are equivalent
$$f(\xi) = 0 \Longleftrightarrow g_1(\xi)=\xi g_2(\xi) = \xi.$$

\vspace{10mm}

\section{}
Show that $\xi = 3\pi$ is a root of $f$ and determine $g_2'(\xi)$.
\vspace{10mm}


\section{}
Determine $L_1 \in (0,1)$such that $\underset{x \in \mathbb{R}}{max}|g_1'(x)|\le L_1$
\vspace{10mm}


\section{}
Show that $3\pi$ is a unique root of the function $f:\mathbb{R} \xrightarrow{} \mathbb{R}$ defined by equation \ref{eq:1a}.
\vspace{10mm}

\section{}
Write a code that you can use to compute the first $k_{max}+1$ approximations $x_k$ and $y_k$ to $3\pi$ from the following two algorithms

\begin{equation}\label{eq:4a}
    \begin{cases}
    x_0 = 100,\\
    x_{k+1}= g_1(x_k), & k=0,1,\dots , k_{max}-1\\
    \end{cases}
\end{equation}

and 

\begin{equation}\label{eq:4b}
    \begin{cases}
    y_0 = 100,\\
    y_{k+1}= g_2(y_k), & k=0,1,\dots , k_{max}-1\\
    \end{cases}
\end{equation}

where $k_{max}$ is an arbitrary positive integer and $g_1$ and $g_2$ are defined by equations \ref{eq:1b} and \ref{eq:1c}, respectively. 

\vspace{10mm}

\section{}
Run your code for $k_{max}=10$ and present a table (Table 1) displaying the iteration number $k \in \{1,2,\dots,10\}$ in the first column, the corresponding approximations $x_k$ defined by equation \ref{eq:4a} in the second column, $y_k$ defined by \ref{eq:4b} in the third column, the corresponding errors of $x_k$ in the fourth column, and the errors of $y_k$ in the fifth column. Use {\fontfamily{qcr}\selectfont \textbf{format long e}}.
\vspace{10mm}

\begin{table}[H]
    \centering
    \begin{tabular}{|c|c|c|c|c|}
    \Xhline{1 pt}
    \textbf{k} & $x_k$& $y_k$& \makecell{errors\\$x_k $ }  & \makecell{errors \\$y_k$}  \\
    \Xhline{2 pt}
    \textbf{1} && &   &  \\
    \Xhline{1 pt}
    \textbf{2} && &   &  \\
    \Xhline{1 pt}
    \textbf{3} && &   &  \\
    \Xhline{1 pt}
    \textbf{4} && &   &  \\
    \Xhline{1 pt}
    \textbf{5} && &   &  \\
    \Xhline{1 pt}
    \textbf{6} && &   &  \\
    \Xhline{1 pt}
    \textbf{7} && &   &  \\
    \Xhline{1 pt}
    \textbf{8} && &   &  \\
    \Xhline{1 pt}
    \textbf{9} && &   &  \\
     \Xhline{1 pt}
    \textbf{10} && &   &  \\
    \Xhline{1 pt}
    \end{tabular}
    \caption{Question 6}
    \label{tab:my_label}
\end{table}


\section{}
Run your code for $k_{max}=2$ and compute approximations $z_k$ defined by

\begin{equation}\label{eq:7}
    \begin{cases}
    z_0 = x_2,\\
    z_{k+1}=g_2(z_k), & k=0,1,\dots , k_{max}-1,
    \end{cases}
\end{equation}

to present a table (Table 2) displaying the iteration number $k \ in \{0,1,2\}$ in the first column,the corresponding approximations $z_k$ defined by \ref{eq:7} in the second  column, and the corresponding errors of $z_k$ in the third column. Use {\fontfamily{qcr}\selectfont \textbf{format long e}}.
\vspace{10mm}

\begin{table}[H]
    \centering
    \begin{tabular}{|c|c|c|}
    \Xhline{1pt}
    \textbf{k} &$z_k$    & \makecell{errors \\ $z_k$}  \\
    \Xhline{2 pt}
    \textbf{0}   & & \\
    \Xhline{1 pt}
    \textbf{1}   & & \\
    \Xhline{1 pt}
    \textbf{2}   & & \\
    \Xhline{1 pt}
    \end{tabular}
    \caption{Question 7}
    \label{tab:q7}
\end{table}



\end{document}